\chapter{Conclusões e Trabalhos Futuros}

O presente trabalho utilizou algoritmos de caminho mínimo em grafos dinâmicos
para determinação de percusos com tempo mínimo em grafos hipotéticos.
A aplicação destes algoritmos resultou na formulação de um modelo computacional em grafos
de topologia dinâmica e atributos dinâmicos. Também foi construída uma ferramenta (\textit{software})
extendida do Dynagraph, que calcula e exibe o caminho mínimo entre dois vértices conhecidos.

Inicialmente foram apresentados os algoritmos de topologia estática e atributos dinâmicos com
limite inferior e superior. Em seguida, esses mesmos algoritmos foram utilizados no algoritmo
de topologia dinâmica e atributos dinâmicos.

Pode-se concluir que o algoritmo utilizado em grafos de topologia dinâmica e atributos dinâmicos
é eficiente ao ponto de garantir a resolução do problema dentro dos objetivos propostos, pois ele
garante um intervalo de previsão para travessia de cada aresta e intervalo para o caminho mínimo.
Mesmo sabendo que o modelo não é o mais eficiente, pois o mesmo é implementado usando o
algoritmo de Dijkstra com Radix Heap, e poderia usar a estrutura de dados Fibonacci heap com a
implementação do radix heap, e com isso reduzir ainda mais a complexidade.

Esta pesquisa também contribuiu com a integração da ferramenta implementada com a ferramenta Dynagraph.
Essa integração permite através do Dynagraph solicitar o caminho mínimo do grafo em análise. Para isso, ele faz 
uma chamada ao \textit{software} deselvolvido. Este, por sua vez, analisa o grafo enviado, calcula o caminho mínimo
e envia para o Dynagraph o mesmo grafo acrescido das arestas do caminho mínimo gerado.

Para trabalho futuros, pretende-se provar que o algoritmo de caminho mínimo para grafos dinâmicos implementado garante que o
intervalo de previsão é mínimo. Também pretende-se otimizar o algoritmo para garantir o cálculo eficiente, pois a pesquisa
somente se concentrou na elaboração do modelo computacional.
Para trabalhos futuros na ferramenta, pretende-se resolver alguns detalhes para garantir o funcionamento da ferramenta.
A aplicação ainda limita-se ao selecionar um grafo em arquivo externo, pois o mesmo seleciona um grafo na implementação.
Pretende-se disponibilizar uma opção na ferramenta que indique o horário inicial de partida
do caminho mínimo. E por fim, aperfeiçoar o Editor de Características extendendo a edição para arestas e integrar ao Dynagraph.
