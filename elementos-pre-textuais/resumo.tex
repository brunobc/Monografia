Este trabalho trata do problema do caminho mínimo para grafos dinâmicos,
onde o custo da travessia de arcos e a mudança de topologia podem ocorrer ao
longo de um horizonte de tempo. Estas situações,por exemplo, aparecem no tráfego dinâmico
e planejamento de rotas para as redes de transporte.
Mostramos os resultados de uma adaptação do método Radix-Heap Dijkstra
para lidar com as diferentes variações de redes dinâmicas, e mostramos nossos resultados
para uma série de grafos dinâmicos selecionados. Usamos o software DYNAGRAPH como um ambiente
computacional para avaliar nossos métodos.
Extendemos o software DYNAGRAPH criando um Editor de Características, que permite
alterar os atributos visuais dos vértices de um grafo dinâmico.

% Separe as palavras-chave por ponto
\palavraschave{Caminho Mínimo. Grafos Dinâmicos}