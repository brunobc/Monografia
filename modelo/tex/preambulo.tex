\documentclass[pnumabnt,normaltoc,capchap,floatnumber=continuous]{abnt}		
\usepackage[bibjustif,abnt-etal-cite=3,abnt-etal-list=0,abnt-full-initials=yes]{abntcite}
\usepackage{modelo/tex/uece}
\usepackage[portuguese,brazilian,portuges]{babel}
\usepackage[utf8]{inputenc}
\usepackage{abnt-alf}
\usepackage{graphicx}
\usepackage{multicol}
\usepackage{listings}
\usepackage{booktabs}
\usepackage{amsmath}
\usepackage{amsthm}
\usepackage{eucal}
\usepackage{amssymb}
\usepackage{mathrsfs}
\usepackage[notintoc,portuguese]{nomencl}
\usepackage[section]{placeins}
\usepackage{algpseudocode,algorithm}
\bibliographystyle{abnt-alf}

\usepackage{xcolor}
\definecolor{verde}{rgb}{0.25,0.5,0.35}
\definecolor{jpurple}{rgb}{0.5,0,0.35}
\definecolor{dkgreen}{rgb}{0,0.4,0}
\definecolor{cinza}{rgb}{0.5,0.5,0.5}
\definecolor{mauve}{rgb}{0.58,0,0.82}


\lstset{
  language=Java,
  basicstyle=\ttfamily\small, 
  keywordstyle=\color{jpurple}\bfseries,
  stringstyle=\color{blue},
  commentstyle=\color{cinza},
  morecomment=[s][\color{blue}]{/**}{*/},
  extendedchars=true,
  showspaces=false,
  showstringspaces=false,
  numberstyle=\tiny\color{gray},
  numbers=left,
  breaklines=true,
  breakautoindent=true, 
  captionpos=b,
  xleftmargin=0pt,
  basicstyle=\ttfamily\scriptsize,
  tabsize=4,
  inputencoding=utf8,
  extendedchars=true,
  literate=%
        {é}{{\'{e}}}1
        {è}{{\`{e}}}1
        {ê}{{\^{e}}}1
        {ë}{{\¨{e}}}1
        {É}{{\'{E}}}1
        {Ê}{{\^{E}}}1
        {û}{{\^{u}}}1
        {ù}{{\`{u}}}1
        {â}{{\^{a}}}1
        {à}{{\`{a}}}1
        {á}{{\'{a}}}1
        {ã}{{\~{a}}}1
        {Á}{{\'{A}}}1
        {Â}{{\^{A}}}1
        {Ã}{{\~{A}}}1
        {ç}{{\c{c}}}1
        {Ç}{{\c{C}}}1
        {õ}{{\~{o}}}1
        {ó}{{\'{o}}}1
        {ô}{{\^{o}}}1
        {Õ}{{\~{O}}}1
        {Ó}{{\'{O}}}1
        {Ô}{{\^{O}}}1
        {î}{{\^{i}}}1
        {Î}{{\^{I}}}1
        {í}{{\'{i}}}1
        {Í}{{\~{Í}}}1
}


% Declaracoes em Português
\algrenewcommand\algorithmicend{\textbf{fim}}
\algrenewcommand\algorithmicdo{\textbf{faça}}
\algrenewcommand\algorithmicwhile{\textbf{enquanto}}
\algrenewcommand\algorithmicfor{\textbf{para}}
\algrenewcommand\algorithmicif{\textbf{se}}
\algrenewcommand\algorithmicthen{\textbf{então}}
\algrenewcommand\algorithmicelse{\textbf{senão}}
\algrenewcommand\algorithmicreturn{\textbf{devolve}}
\algrenewcommand\algorithmicfunction{\textbf{função}}

% Rearranja os finais de cada estrutura
\algrenewtext{EndWhile}{\algorithmicend\ \algorithmicwhile}
\algrenewtext{EndFor}{\algorithmicend\ \algorithmicfor}
\algrenewtext{EndIf}{\algorithmicend\ \algorithmicif}
\algrenewtext{EndFunction}{\algorithmicend\ \algorithmicfunction}

% O comando For, a seguir, retorna 'para #1 -- #2 até #3 faça'
\algnewcommand\algorithmicto{\textbf{até}}
\algrenewtext{For}[3]%
{\algorithmicfor\ #1 $\gets$ #2 \algorithmicto\ #3 \algorithmicdo}

\renewcommand{\listalgorithmname}{Lista de Algoritmos}

\usepackage[T1]{fontenc}
\errorstopmode

\hyphenation{com-pu-ta-ção}
\hyphenation{com-bi-na-tó-ria}

\renewcommand{\nomname}{Lista de Siglas}
\renewcommand{\nomlabel}[1]{\hfil #1\hfil}

\setcounter{secnumdepth}{3}
\setcounter{tocdepth}{3}

\setlength{\topmargin}{-0.3in}

\local{Fortaleza - Ceará}
\cidade{Fortaleza}
\data{2015}

% Informações institucionais
\centro{Centro de Ciências e Tecnologia}
\curso{Graduação em Ciência da Computação}
\cursosimples{Ciência da Computação}
\instituicao{Universidade Estadual do Ceará}

% Descrição para folha de rosto
\comentario{
Monografia apresentada no Curso de \ABNTcursodata\ do \ABNTcentrodata\ da
\ABNTinstituicaodata, como requisito parcial para obtenção do grau de Bacharel
em \UECEcursosimples. 
}

% Descrição um pouco simplificada (removendo o nome do curso ¬¬)
\comentariosimplificado{
Monografia apresentada no Curso de \ABNTcursodata\ do \ABNTcentrodata\ da
\ABNTinstituicaodata, como requisito parcial para obtenção do grau de Bacharel.
}
