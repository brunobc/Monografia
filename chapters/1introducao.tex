\chapter{Introdução}

O problema de caminho mínimo é um dos problemas fundamentais da computação assim como é um problema
clássico em otimização combinatória.
Ele é intensamente estudado e utilizado em diversas áreas como Engenharia de Transportes, Pesquisa Operacional, Ciência
da Computação e Inteligência Artificial. Isso acontece porque tem potencial de aplicação a
inúmeros problemas que ocorrem em transportes, logística, redes de computadores de
telecomunicações, etc \cite{peer}.

O roteamento de veículos em um sistema de transporte é atualmente uma das aplicações mais comuns.
Neste contexto, vértices representam cruzamento de ruas, os arcos representam as vias
e os pesos representam medida de custo, tempo ou distância. O caminho mínimo entre dois cruzamentos
é dado pelo conjunto de arcos que resulta no custo mínimo do percurso.
Este custo não necessariamente é a menor distância a percorrer,
o conceito é mais genérico, considerando algum atributo quantificável, como,
por exemplo, distância, tempo, risco, etc \cite{boaventura}, \cite{cormen}, \cite{ziviani}.
Através do crescente desenvolvimento dos computadores pessoais ou portáteis, como sistemas
de navegação de carros e sistemas embarcados GPS, esse tipo de problema de caminho mínimo
tem se tornado cada vez mais usado.

Diante o grande aumento do tráfego de veículos, se torna indispensável o uso de Sistemas Inteligentes
de Transporte (ITS - $Intelligent Transportation System$), que realizam o controle e gerenciamento desse fluxo de veículos,
permitindo assim o aumento do poder de decisão para o planejamento de ações de maneira
mais inteligente e eficiente.
Sistemas de previsão de tráfego são como forma de auxiliar o controle de tráfego de veículos.
Através de análises de dados históricos e de tempo real sobre
os dados lidos de fluxo de veículos, informações futuras geradas através de modelos
estatísticos e computacionais podem prever o comportamento do tráfego em determinadas
vias \cite{leonard}.

\section{Motivação}

Um sistema que selecione a melhor rota para os condutores ajudaria a diminuir o tráfego intenso em determinados locais.
Todas essas informações seriam geradas em tempo real e poderia usar Sistemas Inteligentes de Transporte, definidos 
como sistemas de transporte que usam tecnologias de informação e
de telecomunicações buscando reduzir congestionamentos e filas.
Entre as inúmeras aplicações, incluem-se sistemas de apoio à navegação em tempo real, cuja finalidade é auxiliar
motoristas a encontrar os melhores caminhos ou rotas para atingirem seus locais de destino.

Inúmeras aplicações baseadas em SIG’s (GIS - $Geographic Information Systems$),
que permitem tratamento computacional a dados geográficos ou geo-referenciados,
vêm sendo disponibilizadas na internet.
São ferramentas em forma de mapa que facilitam o usuário localizar um endereço,
encontrar o menor caminho entre dois lugares, ou até mesmo o caminho mais rápido para chegar ao destino \cite{leonard}.

O uso de sistemas ``on-line'' que sugerem a melhor rota de um ponto de origem a um ponto de destino conhecido,
obtém dados em tempo real, e com isso realizam a previsão de rotas otimizadas, levando em conta o tráfego
dentro das diversas faixas de horário, os usuários, etc.
Os usuários que tiverem acesso ao sistema poderiam saber onde há congestionamentos ou qualquer tipo de obstrução das vias.
Isso acontece porque o sistema recebe informações dos veículos, logo ele é considerado um sistema dinâmico,
pois ele se adapta de acordo com os acontecimentos ao longo do tempo.
Para isso, é necessário um algoritmo de caminho mínimo otimizado que forneça uma resposta para o usuário compatível
com o trajeto que irá realizar.

\section{Objetivos}

\subsection{Objetivo Geral}
Criar uma ferramenta que propõe rotas otimizadas entre dois pontos conhecidos ao longo do tempo
numa rede de topologia dinâmica, utilizando o software Dynagraph \cite{dynagraph}.

\subsection{Objetivos Específicos}
\begin{itemize}
 \item Seguir o modelo computacional Dynagraph \cite{dynagraph} para redes dinâmicas para criacão de um modelo composto
 que aborde redes de topologia estática e dinâmica;
 \item Desenvolver uma ferramenta capaz de sugerir um trajeto com menor tempo
 de percurso entre dois pontos conhecidos, baseada no tempo médio de percurso em trechos
 intermediários, numa rede que pode ser alterada ao longo do tempo.
\end{itemize}

\section{Metodologia de Desenvolvimento}

Para o desenvolvimento dessa solução, os grafos utilizados neste trabalho são hipotéticos,
pois a pesquisa se concentra na modelagem do algoritmo de Dijkstra com Radix Heap aplicado a Grafos Dinâmicos.
O trabalho foi dividido em 3 etapas, que são descritas à seguir:

\begin{itemize}
 \item Analisar modelos de Grafos dinâmicos: determinar dentre os modelos existentes o que melhor
 se adapta ao problema proposto;

 \item Empregar modelos de caminhos mínimos: adaptar a aplicação para sistemas que usam 
 previsão dos tempos de percurso;

 \item Efetuar testes: elaborar relatórios através de testes que possam ser analisados pelo
 software de visualização e edição Dynagraph.
\end{itemize}

\section{Organização do Trabalho}

Este trabalho está organizado em cinco capítulos:
O Capítulo 1 apresenta soluções tecnológicas para resolução de problemas de caminho mínimo,
além das motivações e metodologia para o desenvolvimento da pesquisa.
O Capítulo 2 apresenta uma revisão bibliográfica sobre redes de Topologia Estática, redes Dinâmicas e
sua geração e manutenção, descrevendo algumas soluções.
O Capítulo 3 apresenta a fundamentação teórica para a compreensão do trabalho desenvolvido, abordando
caminhos em grafos estáticos e dinâmicos, e descrevendo os algoritmos desenvolvidos nesta pesquisa.
O Capítulo 4 apresenta os resultados obtidos pelos testes realizados no Dynagraph.
No Capítulo 5 são apresentadas as conclusões deste trabalho e propostas para trabalhos futuros.
