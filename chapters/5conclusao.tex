\chapter{Conclusão e Trabalhos Futuros}

O presente trabalho utilizou algoritmos de caminho mínimo em grafos dinâmicos
para determinação de percusos com tempo mínimo em grafos hipotéticos.
A aplicação destes algoritmos resultou na formulação de um modelo computacional em grafos
de topologia dinâmica e atributos dinâmicos. Também foi construído uma ferramenta ($software$)
extendida do Dynagraph, que exibe o caminho mínimo entre dois pontos conhecidos.

Inicialmente foram apresentados os algoritmos de topologia estática e atributos dinâmicos com
limite inferior e superior.
Dentre os algoritmos implementados, o algoritmo que altera o custo de previsão antes de ultrapassar o tempo
de previsão foi utilizado no algoritmo de topologia dinâmica e atributos dinâmicos.

Pode-se concluir que o algoritmo utilizado em grafos de topologia dinâmica e atributos dinâmicos
é eficiente ao ponto de garantir a resolução do problema dentro dos objetivos propostos, pois ele
garante um intervalo de previsão para travessia de cada aresta e intervalo para o caminho mínimo.
Mesmo sabendo que o modelo não é o mais eficiente, pois o mesmo é implementado usando o
algoritmo de Dijkstra com Radix Heap, e poderia usar a estrutura de dados Fibonacci heap com a
implementação do radix heap, e com isso reduzir ainda mais a complexidade.

Conclui-se também, a integração com a ferramenta Dynagraph enviando o grafo
com o caminho mínimo diretamente, evitando baixar o arquivo JSON. Caso o usuário deseje baixar, ele terá a opção.

Para trabalhos futuros, pretende-se resolver alguns detalhes para garantir o funcionamento da ferramenta.
A aplicação ainda limita-se em determinar o caminho mínimo atráves do primeiro ponto até o último
ponto do grafo, pois não é possivel selecionar os vértices origem e destino. Outra limitação é a
seleção de um grafo em arquivo externo, pois o mesmo seleciona um grafo na implementação.
Pretende-se disponibilizar uma opção na ferramenta que indique o horário inicial de partida
do caminho mínimo. E por fim, aperfeiçoar o Editor de Características extendendo a edição para arestas.
